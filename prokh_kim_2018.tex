\documentclass[12pt,reqno]{report}
\usepackage[utf8]{inputenc}
\usepackage[english,russian]{babel}
\usepackage{amsmath,amsfonts,amssymb}
\usepackage{graphicx}
\usepackage{amsbib}
\usepackage{subcaption}
\usepackage{mathrsfs}
\usepackage{tikz}
\usepackage{float}

\usepackage{graphicx,color}
\usepackage{cite}

\long\gdef\COMMENT#1{}

\definecolor{red}{rgb}{0.8, .0, .0}
\definecolor{blue}{rgb}{.0, 0.0, 0.9}
\def\blue{\color{blue}}
\def\red{\color{red}}
\textheight 24cm \textwidth 17cm \topmargin -2cm \oddsidemargin
0.5cm
\renewcommand{\baselinestretch}{1.25}
\def\sgn{\mathrm{sgn}}

\newtheorem{lemma}{Лемма}
\newtheorem{theorem}{Теорема}
\newtheorem{definition}{Определение}
\newtheorem{corollary}{Следствие}
\newtheorem{proposition}{Предложение}
\newtheorem{problem}{Задача}
\newtheorem{remark}{Замечание}
\renewcommand{\abstractname}{*}




\renewcommand{\thesection}{\arabic{section}.}
\renewcommand{\thedefinition}{\arabic{definition}.}
\renewcommand{\thelemma}{\arabic{lemma}.}
\renewcommand{\thetheorem}{\arabic{theorem}.}
\renewcommand{\thecorollary}{\arabic{corollary}.}
\renewcommand{\theremark}{\arabic{remark}.}
\renewcommand{\theequation}{\arabic{equation}}



%\usepackage[landscape,pdftex]{geometry}

%\textheight 17cm \textwidth 22cm \topmargin -2cm \oddsidemargin
%.5cm

\begin{document}
	
	
	
	{\bf УДК} 517.958 \vskip0.2cm
	
	
	\begin{center}
		{\bf \large Начально-краевая
			задача для уравнения переноса излучения 
			со смешанными  
			условиями сопряжения }\footnote[1]{Работа выполнена при финансовой
			поддержке Российского научного фонда (проект \No 14-11-00079).}
	\end{center}
	\begin{center}
		{\bf А. Ким$^{a,b}$, И.В. Прохоров$^{a,b}$ }
	\end{center}
	\begin{center}
		{\it ${}^a$690041 Владивосток, ул. Радио, 7, Институт прикладной
			математики ДВО РАН; \\ ${}^b$690950 Владивосток, ул. Суханова, 8,
			Дальневосточный федеральный университет}\\
		e-mail: prokhorov@iam.dvo.ru
	\end{center}
	\begin{quote}
		{\bf Аннотация.} Рассматривается задача Коши для нестационарного
		уравнения переноса излучения в трехмерной многокомпонентной среде
		с обобщенными условиями сопряжения, описывающими френелевское и диффузное
		отражение и преломление на границе раздела сред. Доказана
		однозначная разрешимость начально-краевой задачи. Построен численный метод Монте-Карло для
		нахождения решения, учитывающий пространственно-временную локализацию источников излучения, проведены вычислительные
		эксперименты.\\
		{\bf Ключевые слова:} интегро-дифференциальное уравнение,
		 задача Коши, френелевские и ламбертовские  условия
		сопряжения, методы Монте-Карло
	\end{quote}
	
	\section{Введение}
	
	\setcounter{equation}{0}
	\setcounter{definition}{0}\setcounter{lemma}{0}\setcounter{theorem}{0}
	\setcounter{corollary}{0}\setcounter{remark}{0}
	
	
Тематика исследований, затрагивающая  теоретические и численные  аспекты   начально-краевых задач для  кинетических уравнений, имеет достаточно давнюю историю (см., например,\cite{1,2,3,4,5}). Тем не менее и в настоящий момент это направление не потеряло свою актуальность. Бурное развитие вычислительной техники последний десятилетий позволяет решать действительно трудоёмкие вычислительные задачи теории переноса излучения. В свою очередь, усложнение моделей, возникающие в связи с этим, новые постановки прямых, обратных и экстремальных задач требуют теоретического обоснования и  создания численных методов для их решения \cite{6,7,8,9,10,11,12,13,14,15,16,17,18}.

Особую роль в теории переноса излучения играют  условия на границах раздела сред, позволяющие описывать ряд физических эффектов, неучтенных в самом уравнении переноса (отражение, преломление, поверхностные источники  и т.д.). Пренебрежение эффектами подобного рода приводит к традиционным условиям сопряжения, типа непрерывной <<склейки>> решения или <<прострела>>\cite{2,6}. В работах \cite{19,20,21,22,23,24,25,26} исследованы  вопросы разрешимости для стационарных краевых задач с обобщенными условиями сопряжения, описывающими  френелевское или диффузное отражение и преломление на границах раздела сред. В \cite{27,28,29,30,31,32,33} рассмотрены  задачи для нестационарных уравнений переноса излучения  с обобщенными  условиями сопряжения в трехмерной ограниченной области и в плоскопараллельной слоистой среде. 

В настоящей работе исследована задача для  нестационарного уравнения переноса излучения в случае, когда оператор сопряжения является линейной комбинацией операторов френелевского и ламбертовского типов. Доказана корректность начально-краевой задачи и предложен метод Монте-Карло для численного  нахождения решения. Ранее в работе \cite{33} была  исследована начально-краевая задача с френелевскими условиями сопряжения на границе раздела сред, предложен метод Монте-Карло  с ветвлением для численного нахождения решения, оценена его трудоемкость. К недостаткам предложенного алгоритма можно отнести  тот факт, что метод обеспечивает неплохую  сходимость только для источников излучения равномерно распределенных в пространстве и времени. В данной работе, помимо включения диффузного отражения и преломления в рассматриваемую модель, нами также была предпринята попытка построения метода, учитывающего априорную информацию о распределнии источников излучения.

Рассмотрим интегродифференциальное уравнение следующего
вида \cite{29,30,31,32,33}
\begin{equation}
\left (\frac{1}{v(r)} \frac{\partial }{\partial t } + \omega
\cdot \nabla_r + \mu(r) \right) I(r,\omega,t)=\sigma(r)
\int\limits_{\Omega} p(r,\omega \cdot
\omega')I(r,\omega',t)d\omega' + J(r,\omega,t).
\end{equation}
Уравнение (1) описывает нестационарный процесс взаимодействия
излучения с веществом, а функция $I(r,\omega,t)$ интерпретируется
как плотность потока частиц в момент времени $t\in [0,\infty)$,
в точке $r\in \mathbb{R}^3$, движущихся со скоростью $v$ в
направлении единичного вектора $\omega \in \Omega=\{\omega \in
\mathbb{R}^3: |\omega|=1\}$. Функции $\mu,\sigma, p$ и $J$ имеют
смысл коэффициентов ослабления и рассеяния,
индикатрисы рассеяния и плотности внутренних источников.

Процесс переноса излучения происходит в
многокомпонентной системе $G$, состоящей из объединения конечного
числа ограниченных и попарно непересекающихся подобластей
$G_1,G_2,...,G_m$, причем замыкание $\overline{G}$ является
выпуклым множеством в $\mathbb{R}^3$. Граница каждой области
$G_i, \partial G_i=\overline{G}_i\setminus G_i$ принадлежит
классу $C^1$. Поверхность $\partial \overline {G}$ будем называть
внешней границей множества $G$, а $\gamma =\partial G \setminus
\partial \overline{G}$ --- внутренней границей множества $G$.
Для определенности, говоря о единичном векторе $n(z)$ нормали к
поверхности $\partial \overline{G}$ в точке $z$, будем иметь
ввиду внешнюю нормаль. Если $z$ является точкой контакта двух
смежных областей $G_i$ и $G_j$, $1 \leq i < j \leq m$, то нормаль
$n(z)$ выбирается внешней к поверхности с большим индексом, то
есть к $\partial{G_j}$.




Для краткости изложения введем ряд обозначений:
$$
X=G\times \Omega \times [0,\infty), \quad X_0=\{(r,\omega,t)\in
X:\; t=0 \}, 
$$
$$
\Omega_{\pm}(z)= \{ \omega \in \Omega : \sgn (n(z) \cdot
\omega)=\pm 1\}, \quad \Gamma^{\pm}_{ext} =
\partial \overline{G} \times \Omega_{\pm}(z), 
$$
$$
 \Gamma_{int}=\gamma \times \Omega, \quad
\Gamma^{\pm}=\Gamma^{\pm}_{ext}\cup \Gamma_{int},
\quad
Y_{int}=\Gamma_{int}\times[0,\infty), \quad
Y^{\pm}_{ext}=\Gamma^{\pm}_{ext}\times [0,\infty), 
$$
$$
X^{-}_{ext}=X_0 \cup Y^{-}_{ext}, \quad  X^{-}=X^{-}_{ext}\cup Y_{int},
\quad X^{+}=Y^{+}_{ext}\cup Y_{int}.
$$
В каждой из подобластей $G_i$ функции $v(r),\mu(r), \sigma(r),
p(r,\omega \cdot \omega')$ не зависят от переменной $r$ и
принимают значения $v(r)=v_i>0, \mu(r)=\mu_i > 0,
\sigma(r)=\sigma_i, \sigma(r) \leq \mu(r)$, $p(r,\omega \cdot
\omega')=p_i(\omega \cdot \omega') \geq 0$, где функция
$p_i(\omega \cdot \omega')$ интегрируема на отрезке $[-1,1]$ и
удовлетворяет условию нормировки
$$
\int \limits_{\Omega} p_i(\omega \cdot \omega') d\omega'=1.
$$
Функция $J$ неотрицательная и принадлежит $L_{\infty}(X)$, где
$L_{\infty}(X)$ пространство вещественных функций измеримых и
почти всюду ограниченных на множестве $X$ с нормой 
$$
\|f\|_{L_{\infty}(X)}=
\mathop{\fam0 ess\,sup} \limits_{x \in X} |f(x)|.
$$
Присоединим к уравнению (1) начальные и граничные условия
\begin{equation}
I^- =h_0(r,\omega) \quad \text{на} \quad X_0,
\end{equation}
\begin{equation}
I^-= h_{ext}(z,\omega,t) \quad \text{на} \quad Y^-_{ext},
\end{equation}
\begin{equation}
I^-=\mathcal BI^+ \quad \text{на}\quad Y_{int}.
\end{equation}
В соотношениях (2),(3) функция $h_0(z,\omega)\geq 0$
характеризует состояние процесса в начальный момент времени $t=0$,
a функция $h_{ext}(z,\omega,t) \geq 0$ имеет смысл плотности
потока излучения входящего в среду $G$. Функции $I^{\pm}$ являются
предельными значениями функции $I$, $I^{\pm}(z,\omega,t)= \lim
\limits_{\epsilon \to -0} I(z \pm \epsilon \omega, \omega,t \pm
\epsilon)$. 


Оператор сопряжения $\mathcal B$ представляет собой линейную комбинацию френелевского и диффузного операторов:
\begin{equation}
\mathcal BI^+ = \alpha_f \mathcal B_{f}I^+ + \alpha_d \mathcal
B_{d}I^+.
\end{equation}
Здесь  $B_{f}$ --  френелевский оператор сопряжения, описывающий зеркальное отражение и преломление по закону Снеллиуса на поверхности раздела сред  \cite{20,22,29,31,33}
\begin{equation}
(\mathcal B_{f}I^+)(z,\omega,t) = R(z,\omega)
I^+(z,\omega_{re},t) + T(z,\omega) I^+(z,\omega_{tr},t),
\end{equation}
а         $B_{d}$ --  оператор, описывающий диффузное отражение и преломление \cite{30,32} на границе раздела  по закону Ламберта  
\begin{multline}
(\mathcal B_d I^+)(z,\omega,t) = \frac{R_d (z,\omega)}{\pi}\int
\limits_{\Omega(z,-\omega)}|n(z)\cdot \omega'| I^+(z,\omega',t)d
\omega' + \\+ \frac{T_d (z,\omega)}{\pi} \int
\limits_{\Omega(z,+\omega)}|n(z)\cdot \omega'| I^+(z,\omega',t) d
\omega'.
\end{multline}
Величины $ \alpha_f(z),\alpha_d(z) \geq 0$, $ \alpha_f +\alpha_d \leq 1$  --- определяют вклад френелевского и диффузного отражения и преломления на  границе раздела сред. При  $\alpha_f +\alpha_d < 1$  поверхность $\gamma$ является частично
поглощающей.

В соотношениях (5),(6) использованы следующие обозначения:   
$$
\omega_{re} =\omega -2 \nu n, \quad \omega_{tr} =\psi (z,\nu) n +
\widetilde{\kappa}(z,\nu)( \omega - \nu n),   \quad
\nu=\nu(z)=\omega \cdot n(z),
$$
$$
 \widetilde{\kappa}(z,\nu) =
  \begin{cases}
 {\kappa_i}/{\kappa_j}, & \text{если} \,    \quad z \in \partial G_i \cap \partial G_j, \;   0<     \nu(z) \leq 1, \\
 {\kappa_j}/{\kappa_i}, & \text{если} \,    \quad z \in \partial G_i \cap \partial G_j, \;   -1 \leq \nu(z)  <
 0,
  \end{cases}
$$
$$
\psi(z,\nu)=
 \begin{cases}
 {\sgn}(\nu) \sqrt{1- \widetilde{\kappa}^2(z,\nu) (1-\nu^2)}, & \text{если}
\;      1- \widetilde{\kappa}^2(z,\nu) (1-\nu^2) \geq 0, \\
\quad  0,     &  \text{иначе},
  \end{cases}
$$
$$
R (z,\omega)=\frac{1}{2} (R^2_{\|}(z,\nu)+R^2_{\bot}(z,\nu)),
\quad
T(z,\omega)= 
    1-R(z,\omega),
$$
$$
R_{\|}(z,\nu)=\frac{\widetilde{\kappa}(z,\nu)
\psi(z,\nu)-\nu}{\widetilde{\kappa}(z,\nu) \psi(z,\nu)+\nu},\quad
R_{\bot}(z,\nu)=\frac{\psi(\nu)-\widetilde{\kappa}(\nu)\nu}
{\psi(z,\nu)+\widetilde{\kappa}(z,\nu)\nu}.
$$
$$
 R_d(z,\omega) =
  \begin{cases}
 R^+_d(z), & \text{если} \,     (z,\omega) \in \gamma \times \Omega_+(z), \\
 R^-_d(z), & \text{если} \,     (z,\omega) \in \gamma \times  \Omega_-(z), \\
  \end{cases}
$$
$$
 T_d(z,\omega) =
  \begin{cases}
 1-R^+_d(z), & \text{если} \,     (z,\omega) \in \gamma \times \Omega_+(z), \\
 1-R^-_d(z), & \text{если} \,      (z,\omega) \in \gamma \times \Omega_-(z),
  \end{cases}
$$
$$
\Omega(z,\omega) =\{\omega' \in \Omega \; |\;  ( \omega
\cdot n(z))(\omega' \cdot n(z))>0 \}, \quad
\Omega_{\pm}(z)=\Omega(z,\pm n(z)).
$$
Предполагается, что  функции $R_{d}^{\pm}(z)$, $T_{d}^{\pm}(z)$ и  $ \alpha_f(z),\alpha_d(z) $ постоянны  при всех $z \in \partial G_i$.
На внешней границе  $\partial \overline{G}$ эффекты отражения,  преломления и поглощения отсутствуют, то есть поверхность  $\partial \overline{G}$ является <<фиктивной>>  границей раздела сред.

\section{Постановка задачи Коши}

Относительно множества $G$ дополнительно будем предполагать
выполнение условия обобщенной выпуклости \cite{2,6}: любая
прямая, имеющая общую точку с $G$, пересекает $\partial G$ в
конечном числе точек.

Пусть $\Pi_{\omega}$ -- ортогональная проекция множества $G$ на
плоскость, перпендикулярную направлению $\omega$ и проходящую
через фиксированную точку в $\mathbb{R}^3$, а множество
$\Pi_{\xi,\omega}$, где $\xi \in \Pi_{\omega}, \omega \in \Omega$
есть пересечение прямой $\{ \xi + \tau \omega, \; - \infty < \tau
< +\infty \}$ и множества $G$. Тогда, по условию обобщенной
выпуклости, одномерное открытое множество $\Pi_{\xi,\omega}$
является объединением конечного числа интервалов
\begin{equation}
\begin{array}{c}
\Pi^i_{\xi,\omega}=\{\xi + \tau \omega, \; \tau_i (\xi,\omega) < \tau
<
\tau_{i+1} (\xi,\omega) \}, \quad i=1,...,q(\xi,\omega), \\
- \infty < \tau_1(\xi,\omega) < \tau_2 (\xi,\omega) < ... <\tau_{q+1}
(\xi,\omega) < + \infty, \\
q(\xi,\omega) \leq \overline{q}= \sup \limits_{(\xi,\omega)\in
	\Pi_{\omega} \times \Omega} q(\xi,\omega) < \infty,\quad
\Pi_{\xi,\omega}= \bigcup \limits^{q(\xi,\omega)}_{i=1}
\Pi^i_{\xi,\omega}.
\end{array}
\end{equation}
Отметим, что
$$
\Gamma^{-}_{ext}= \{\xi + \tau_1(\xi,\omega) \omega, \; \xi \in
\Pi_{\omega} \} \times \Omega, \quad \Gamma^{+}_{ext}= \{\xi +
\tau_{q+1}(\xi,\omega) \omega, \; \xi \in \Pi_{\omega} \} \times
\Omega,
$$
$$
\begin{array}{rl}
\Gamma^{-}&= \{\xi + \tau_i(\xi,\omega) \omega, \; \xi \in
\Pi_{\omega},\; i=\overline{1,q(\xi,\omega)} \} \times \Omega,
\\
\Gamma^{+}&= \{\xi + \tau_{i+1}(\xi,\omega) \omega, \; \xi \in
\Pi_{\omega}, \;i=\overline{1,q(\xi,\omega)} \} \times \Omega.
\end{array}
$$
Для удобства введем в рассмотрение функцию $h \in L_{\infty}( X^{-})$ 
по формуле:
$$
h(z,\omega,t)=\left\{%
\begin{array}{ll}
h_0(z,\omega), & \hbox{если} \; (z,\omega,t) \in X_0, \\
h_{ext}(z,\omega,t), & \hbox{если} \; (z,\omega,t) \in Y^-_{ext}, \\
0,                   & \hbox{если} \; (z,\omega,t) \in Y_{int}, \\
\end{array}%
\right.
$$
и доопределим оператор сопряжения, полагая $({\cal B} \phi) (r,\omega,t)=0$ для всех  $(r,\omega,t) \in X^-_{ext}$ .
Таким образом функция $h$ и функции, принадлежащие области определения оператора ${\cal B}$, заданы на одном и том же множестве $X^{-}$.

Пусть $d(r,-\omega,t)=\min\{d(r,-\omega), v_i t  \}$, где  величина $d(r,-\omega)$ --- расстояние от точки $r \in G_i \subset G$ до
границы области $G_i$ в направлении вектора $-\omega$, то есть
$d(r,-\omega) = \sup \limits_{\tau
	> 0} \{ r- \tau' \omega \in G_i \; \text{для любого} \; \tau' \in [0,\tau) \}$.

{\it Будем говорить}, что функция $f(r,\omega,t)$ принадлежит
$D(X)$, если после надлежащего изменения на множестве меры нуль в
$X$:

1) при почти всех $(r,\omega,t) \in G_i \times \Omega
\times [0,\infty)$, $i=1,...,m$, функция $f(r+\tau\omega,\omega,t+\tau/v_i)$
абсолютно непрерывна по $\tau,\,\tau \in
(d(r,-\omega,t), d(r,\omega)]$, причем, производная   функции $f$ в точке $(r,t)\in G_i \times [0,\infty)$  в направлении вектора $(\omega_1,\omega_2,\omega_3,1/v_i)$
$$ 
\left (\frac{1}{v_i} \frac{\partial }{\partial t } + \omega
\cdot \nabla_r \right) f(r,\omega,t)= \left.
\frac{\partial}{\partial \tau}
f\left(r+\tau\omega,\omega,t+\tau/v_i\right) \right |_{\tau=0},
$$
существующая   при почти всех  $(r,\omega,t) \in X$, принадлежит  пространству $L_{\infty}
(X)$;

2) $f \in L_{\infty} (X)$, $f^{\pm}\in L_{\infty} (X^{\pm})$;



Пусть операторы ${\cal L}: D(X) \to L_{\infty}(X)$ и ${\cal S}:
L_{\infty}(X) \to L_{\infty}(X)$ определены равенствами
\begin{equation}
{\cal L} f = \left (\frac{1}{v} \frac{\partial }{\partial t } +
\omega \cdot \nabla_r \right) f + \mu f,
\end{equation}
\begin{equation}
{\cal S}f= \sigma(r)
\int\limits_{\Omega} p(r,\omega \cdot
\omega')f(r,\omega',t)d\omega',
\end{equation}
тогда {\it решением начально-краевой задачи (1)-(4)} будем
называть функцию $I \in D(X)$, удовлетворяющую уравнению
\begin{equation}
{\mathcal L} I = {\mathcal S}I + J \quad \text{почти всюду на} \quad X
\end{equation}
и условию
\begin{equation}
I^-= {\mathcal B}I^+ + h \quad \text{почти всюду на} \quad X^-.
\end{equation}



\section{Исследование разрешимости начально-краевой задачи}


Введем в рассмотрение операторы $\mathcal P: L_{\infty}(X^-) \to D(X)$ и $\mathcal E:  L_{\infty}(X) \to D(X)$ следующими формулами:
$$
(\mathcal P
\phi)(r,\omega,t)=\phi^-(r-d(r,-\omega,t)\omega,\omega,t-d(r,-\omega,t)/v_i)
\exp \left(- \mu_i d(r,-\omega,t)\right), \quad
r\in G_i,
$$
$$
(\mathcal E\Phi)(r,\omega,t)=\int \limits^{d(r,-\omega,t)}_0 \exp
\left(- \mu_i \tau\right) \Phi(r-\tau
\omega,\omega,t-\tau/v_i)d\tau.
$$
Так как ${\mathcal L} {\mathcal P} \phi =0$, ${\mathcal P}
\phi|_{X^-} =\phi^-$ и ${\mathcal L} {\mathcal E} \Phi = \Phi$,
${\mathcal E} \Phi|_{X^-}=0$, то разрешимость начально-краевой
задачи в $D$ эквивалентно разрешимости уравнения
\begin{equation}
I= \mathcal P(\mathcal B I^+ +h ) + \mathcal E ( \mathcal S I +
J).
\end{equation}


Предметом исследования  этого параграфа краевая задача для  вспомогательного уравнения 
\begin{equation}
\mathcal Lf =\Phi, \qquad \Phi \in L_{\infty}(X), \; f \in D(X)
\end{equation}
c граничным условием (12). Из (13) вытекает, что решение этой задачи эквивалентно решению соответствующего интегрального уравнения 
\begin{equation}
f= \mathcal P (h+\mathcal B f^+) +{\mathcal E}\Phi.
\end{equation}
В дальнейшем нам понадобится дополнительное ограничение на регулярность границы множества $G$: существуют такие $\delta,\delta',\, 0< \delta < 1,\, \delta'>0$, что для любой точки $z \in \partial G$ множество 
$\{z+\tau\omega:\, \delta \leq |n(z)\cdot \omega| \leq 1,\, 0 <|\tau| <\delta'\}$ не имеет пересечения с $\partial G$. 
Условия подобного рода  типичны при исследовании свойств анизотропных соболевских пространств. 



\begin{lemma}
Уравнение (14) 	  имеет в $D$ не более одного решения.
\end{lemma}
{\bf Доказательство.}
Пусть функции $f_1,f_2\in D$ -- решения неоднородного уравнения (14), тогда функция  $f=f_1-f_2$ удовлетворяет однородному уравнению (14). Следовательно, функция $f$ является решением уравнения  (14) при $\Phi=0$, а $f^+$ удовлетворяет  соотношению
\begin{equation}
f^+= \mathcal P (\mathcal B f^+)\; \text{на} \; X^+.
\end{equation}
Покажем, что $f^+=0$ почти всюду на $X^+$. 
Так как $R+T \leq 1, R_d+T_d \leq 1$, то  операторы сопряжения,  описывающие
френелевское и ламбертовское  отражение на границе среды $G$, удовлетворяют условиям  
$\|\mathcal B_f \|_{L_{\infty}(X^-)\to L_{\infty}(X^+)}\leq 1 $, $\|\mathcal B_d \|_{L_{\infty}(X^-)\to L_{\infty}(X^+)} \leq 1$. Учитывая это обстоятельство,  из (16) получаем
\begin{multline}
\|\mathcal B f^+\|_{L_{\infty}(X^-)} = \|\mathcal B f^+\|_{L_{\infty}(X^-\setminus X^-_{ext})} = \| \mathcal B \mathcal P \mathcal B  f^+\|_{L_{\infty}(X^-\setminus X^-_{ext})} \leq \\ \leq  \| \alpha_f \mathcal B_f \mathcal P   +\alpha_d \mathcal B_d \mathcal P\|_{L_{\infty}(X^-\setminus X^-_{ext})\to L_{\infty}(X^+)} \|  \mathcal B f^+\|_{L_{\infty}(X^-)} \leq \\ \leq \left \| \alpha_f  +\alpha_d \|\mathcal B_d  \mathcal P \|_{L_{\infty}(X^-\setminus X^-_{ext})\to L_{\infty}(X^+)} \right\|_{L_{\infty}(\gamma)} \|   f^+\|_{L_{\infty}(X^+)}.
\end{multline}
Оценим норму  оператора
$\mathcal B_d \mathcal P: L_{\infty}(X^-\setminus X^-_{ext})\to L_{\infty}(X^+)$.
Так как на множестве $X^- \setminus X^-_{ext}$ функция $d(r,-\omega,t)$ равна $d(r,-\omega)$,  
и операторы $\mathcal B_d$ и $\mathcal P$ неотрицательные, то получаем следующее неравенство \begin{multline}
\|\mathcal B_d \mathcal  P  \|_{L_{\infty}(X^-\setminus X^-_{ext})\to L_{\infty}(X^+)} \leq \\ \leq  \sup \limits_{(z,\omega)\in \Gamma^+} \left| \frac{R_d (z,\omega)}{\pi}  \int
\limits_{\Omega(z,-\omega)} e^{-\overline{\sigma} d(z,-\omega')} |n(z)\cdot \omega'| d \omega' +   \frac{T_d (z,\omega)}{\pi} \int
\limits_{\Omega(z,+\omega)} e^{-\overline{\sigma} d(z,-\omega')} |n(z)\cdot \omega'| d
\omega'  \right|,
\end{multline}
где $\overline{\sigma}=\max \limits_{i=\overline{1,m}} \sigma_i$.
Обозначим через $W_{\pm}$ следующие интегралы
$$
W_{\pm}(z,\omega)=\frac{1}{\pi} \int \limits_{\Omega(z,\pm \omega)}\exp \left (- \overline{\sigma} d(z,-\omega')\right)  |n(z)\cdot \omega'| d\omega'
$$
и для всех $n(z) \cdot \omega>0$ оценим функцию $W_+(z,\omega)$.
Введем лебегово интегрирование на сфере $\Omega$, полагая $d\omega'=d\nu d\varphi$, где:
\begin{equation}
\begin{cases}
\nu = n(z) \cdot \omega', & -1 \leq \nu \leq 1, \\
\varphi= \arctg(\omega'_2/\omega'_1), & 0 \leq \varphi < 2 \pi,
\end{cases}
\begin{cases}
\omega'_1=&\cos (\varphi) \sqrt{1-\nu^2}, \\
\omega'_2=&\sin (\varphi) \sqrt{1-\nu^2}, \\
\omega'_3=&\nu.
\end{cases}
\end{equation}
Принимая во внимание  условие на регулярность границы, получаем 
\begin{multline}
W_+(z,\omega)=\frac{1}{\pi} \int \limits_{0}^{2\pi} \int \limits_{0}^{1} \exp \left (- \overline{\sigma} d(z,-\omega'(\nu,\varphi))\right)  \nu d \nu d\varphi \leq \\  \leq \frac{1}{\pi} \int \limits_{0}^{2\pi} \int \limits_{0}^{\delta}   \nu d \nu d\varphi
+ \frac{1}{\pi} \int \limits_{0}^{2\pi} \int \limits_{\delta}^{1} \exp \left (- \overline{\sigma}\delta'\right)  \nu d \nu d\varphi
\leq \\ \leq
\delta^2 +  (1-\delta^2)\exp \left (- \overline{\sigma} \delta'\right) =1-(1-\delta^2)(1-\exp \left (- \overline{\sigma} \delta'\right))
\end{multline}
Нетрудно видеть, что при величина $C_1=1-(1-\delta^2)(1-\exp \left (- \overline{\sigma} \delta'\right))$ меньше единицы  при    $0<\delta<1,\,\delta'>0$.
Аналогично показывается, что   $W_+(z,\omega) \leq C_1$ при  $n(z) \cdot \omega<0$ и $W_-(z,\omega) \leq C_1$ при  почти всех $(z,\omega)$. Таким образом, из (18) и (20) находим
\begin{equation}
\|\mathcal B_d \mathcal  P  \|_{L_{\infty}(X^-\setminus X^-_{ext})\to L_{\infty}(X^+)}  \leq    
C_1 \sup \limits_{(z,\omega) \in \Gamma^+}\left| R_d (z,\omega)+ T_d (z,\omega) \right| \leq C_1<1.
\end{equation}
Из соотношений (16), (21) получаем
\begin{multline}
\|f^+\|_{L_{\infty}(X^+)} \leq \|\mathcal P \mathcal B f^+\|_{L_{\infty}(X^+)} 
\leq 
\| \mathcal B f^+\|_{L_{\infty}(X^-)} 
\leq \\ \leq 
\|\alpha_f +\alpha_d \|\mathcal  B_d \mathcal P \|_{L_{\infty}(X^-\setminus X^-_{ext})\to L_{\infty}(X^+)} \|_{L_{\infty}(\gamma)}  \|  f^+\|_{L_{\infty}(X^+)} \leq \\ \leq 
 \|\alpha_f  +\alpha_d C_1 \|_{L_{\infty}(\gamma)}  \| f^+\|_{L_{\infty}(X^+)}= C_2\| f^+\|_{L_{\infty}(X^+)}.
\end{multline}
По условию функция $\alpha_d(z)$  постоянна при $z\in \partial G_i$, $i=1,...,m$, следовательно, если $\alpha_d(z)>0$ хотя бы на одной границе $z\in \partial G_j$, то  константа $C_2= \|\alpha_f  +\alpha_d C_1 \|_{L_{\infty}(\gamma)}$ в (22) меньше единицы. Так как $C_2<1$ , то неравенство (22) возможно только при  $f^+=0$ почти всюду на $X^+$ и, следовательно,  из (15) вытекает, что и $f=0$  почти всюду на $X$. 

Рассмотрим случай $\alpha_d(z)=0$ всюду на $\gamma$. В этом случае оператор сопряжения $\mathcal B$ включает только  френелевскую составляющую  $\mathcal B_f$.

Пусть функции $f_1$ и $f_2$ из $D$
удовлетворяют уравнению (14) и $\kappa(r)=\kappa_i,\,r \in G_i$ --
показатель преломления среды $G$. Тогда функция
$f=(f_1-f_2)/\kappa^2$ удовлетворяет однородному уравнению
${\mathcal L} f=0$ с граничными условиями вида:
\begin{equation}
f^-(z,\omega,t)=R (z,\nu) f^+(z,\omega_{re},t) +
\widetilde{\kappa}^2(z,\nu) T(z,\nu) f^+(z,\omega_{tr},t), \;
(z,\omega,t) \in Y_{int},
\end{equation}
\begin{equation}
f^-(z,\omega,t)=0, \quad (z,\omega,t) \in X^-_{ext}.
\end{equation}
Не умаляя общности, для удобства
проведения доказательства леммы доопределим функцию $f$ при $t<0$
нулем и запишем решение уравнения ${\mathcal L}f=0$, справедливое при
почти всех $(\xi,\tau,\omega,t) \in \Pi_{\omega} \times
\Pi_{\xi,\omega} \times \Omega \times [0,\infty)$
$$
f(\xi+\tau\omega,\omega,t+\tau/\tilde{v}_i)=f^-(\xi+\tau_i (\xi,\omega)
\omega,\omega, t+\tau_i (\xi,\omega)/\tilde{v}_i) \exp \left ( - \int
\limits^{\tau}_{\tau_i(\xi,\omega)} \mu(\xi+\tau'\omega) d\tau'
\right ),
$$
где величина $\tilde{v}_i$ в аргументе функции  $f(\xi+\tau\omega,\omega,t+\tau/\tilde{v}_i)$ принимает значение $v_j$ если
интервал $\{\xi+\tau\omega$, $\tau\in (\tau_i(\xi,\omega), \tau_{i+1}(\xi,\omega))\}$ принадлежит области $G_j$. 



Из последнего соотношения непосредственно вытекает, что функция
$f(\xi+\tau\omega,\omega,t+\tau/v)$ при $\tau \in
(\tau_i(\xi,\omega), \tau_{i+1}(\xi,\omega)]$ не меняет знака. 
Учтем также тот факт, что ${\mathcal L f} = \Phi \in L_{\infty}(X)$ и,
следовательно, по теореме Фубини почти всюду при $t\in[0,\infty)$
функция ${\mathcal L f}$ является ограниченной и измеримой на
множестве $G\times \Omega$.
Имея это
ввиду, умножим уравнение ${\mathcal L}f=0$ на функцию $ {\sgn}
(f(r,\omega,t))$ и проинтегрируем на множестве $G\times \Omega$
\begin{multline}
A=\int \limits_{\Omega} \int \limits_G \left \{ {\sgn}
(f(r,\omega,t))\left(\frac{1}{v(r)} \frac{\partial
	f(r,\omega,t)}{\partial t} +\omega \cdot \nabla_r f(r,\omega,t) +
\mu (r) f(r,\omega,t)\right)\right \}
dr d\omega=
\\
= \int \limits_{\Omega} \int \limits_{\Pi_{\omega}} \sum
\limits^{q(\xi,\omega)}_{i=1} \int \limits_{\Pi^i_{\xi,\omega}}
\frac{\partial |f(\xi+\tau\omega,\omega,t+\tau/\tilde{v}_i)|}{\partial \tau}
d\tau d \xi d \omega +\int \limits_{\Omega} \int \limits_G \mu (r)
|f(r,\omega,t)|
dr d\omega= \\
=
\int \limits_{\Omega} \int \limits_{\Pi_{\omega}}\{
|f|^+(\xi+\tau_{q+1}
(\xi,\omega) \omega,\omega,t+\tau_{q+1}(\xi,\omega)/\tilde{v}_{q+1}) -
|f|^-(\xi+\tau_{1} (\xi,\omega) \omega,\omega,t+\tau_{1}(\xi,\omega)/\tilde{v}_{1})
+
\\+
\sum \limits^{q(\xi,\omega)}_{i=2}
(|f|^+ - |f|^-)(\xi+\tau_i (\xi,\omega)
\omega,\omega, t+\tau_i (\xi,\omega)/\tilde{v}_i)\} d\xi d\omega + \int
\limits_{\Omega} \int \limits_G \mu(r) |f(r,\omega,t)| dr d\omega
= 0.
\end{multline}
Учитывая граничные условия (23),(24), и соотношение (25), по
теореме о замене переменных в поверхностном интеграле ($ d\xi =
|n(z) \cdot \omega| d s_z$), получаем:
\begin{multline}
A =
\int \limits_{\partial \overline{G}} \int \limits_{\Omega_-(z)}
|f(z,\omega,t)|^+ |\nu | d \omega d s_z
+
\\+
\int \limits_{\gamma} \int \limits_{\Omega} \left
\{|f(z,\omega,t)|^+ - |R(z,\nu ) f^+(z,\omega_{re},t) +
\widetilde{\kappa}^2(z,\nu )T(z,\nu )
f^+(z,\omega_{tr},t) ) | \right\} |\nu |d\omega d s_z +\\
+ \int \limits_{\Omega} \int
\limits_G \mu(r) |f(r,\omega,t)| dr d\omega =A_1 +A_2 +A_3, \quad
\nu = n(z)\cdot \omega,
\end{multline}
где через $A_1, A_2, A_3$ обозначены соответствующие слагаемые в
соотношении (24). Сразу заметим, что
\begin{equation}
A_1 = \int \limits_{\partial \overline{G}} \int
\limits_{\Omega_-(z)}
|f(z,\omega,t)|^+ |\nu | d \omega d
s_z \geq 0, \quad \nu = n(z)\cdot \omega.
\end{equation}
Так как $R, T \geq 0$, то
\begin{multline}
|f(z,\omega,t)|^+ - |R(z,\nu ) f^+(z,\omega_{re},t) +
\widetilde{\kappa}^2(z,\nu )T(z,\nu ) f^+(z,\omega_{tr},t) | \geq \\
\geq | f (z,\omega,t)|^+ - R(\nu ) |f(z,\omega_{re},t)|^+ -
\widetilde{\kappa}^2(z,\nu )T(\nu ) |f(z,\omega_{tr},t) |^+,
\end{multline}
следовательно,
\begin{multline}
A_2 \geq \int \limits_{\gamma} \int \limits_{\Omega}| f
(z,\omega,t)|^+|\nu | d\omega d s_z -   \int \limits_{\gamma} \int \limits_{\Omega}  R(z,\nu ) |f(z,\omega_{re},t)|^+ |\nu |d\omega d s_z - \\- 
 \int \limits_{\gamma} \int \limits_{\Omega} \widetilde{\kappa}^2(z,\nu )T(z,\nu ) |f(z,\omega_{tr},t) |^+ 
|\nu |d\omega d s_z = A_{2,1}- A_{2,2} - A_{2,3}.
\end{multline}

Поскольку вектора $\omega,\omega_{re},\omega_{tr}$ лежат в
одной плоскости и 
$$
\nu_{re}=n(z) \cdot \omega_{re} = - n(z) \cdot \omega
=-\nu,\qquad \nu_{tr}=n(z)\cdot\omega_{tr}=\psi(z,\nu),
$$
то после параметризации сферы вида (19) для интегралов $A_{2,2}$,  $A_{2,3}$ получим следующие выражения
$$
 A_{2,2}= \int \limits_{\gamma} \int
\limits^{2\pi}_{0} \int \limits^{1}_{-1} R(z,\nu)
|f(z,\omega(-\nu,\varphi),t)|^+ |\nu| d \nu d \varphi  d s_z,
$$
$$
 A_{2,3} = \int \limits_{\gamma}
 \int
\limits^{2\pi}_{0} \int \limits^{1}_{-1} \widetilde{\kappa}^2(z,\nu )
T(z,\nu) |f(z,\omega(\psi(\nu),\varphi),t)|^+ |\nu| d \nu d
\varphi  d s_z.
$$
В интеграле $ A_{2,2}$  сделаем замену $\nu'=-\nu$, а в интеграле $A_{2,3}$ --- замену $\nu'=\psi(z,\nu)$.
Так как знаки функций $\nu'=\psi(z,\nu)$ и $\nu$ совпадают, то
$\widetilde{\kappa}(z,\nu')=\widetilde{\kappa}(z,\nu)$. Следовательно,
$$
\psi'(z,\nu)=\dfrac{{\sgn}(z,\nu) \cdot \nu\cdot
	\widetilde{\kappa}^2(z,\nu)}{\sqrt{1-\widetilde{\kappa}^2(z,\nu)(1-\nu^2)}}
= \dfrac{ \nu \cdot\widetilde{\kappa}^2(z,\nu)}{\psi(z,\nu)} = \\=  \dfrac{
	\nu \cdot\widetilde{\kappa}^2(z,\nu)}{\nu'}= \dfrac{ \nu
	\cdot\widetilde{\kappa}^2(z,\nu')}{\nu'}
$$
и
$$
|\nu|d\nu= \frac{|\nu| \cdot \nu'
	d\nu'}{\nu\cdot\widetilde{\kappa}^2(z,\nu')} = \frac{{\sgn}(\nu)
	\cdot \nu' d\nu'}{\widetilde{\kappa}^2(z,\nu')}= \\= \frac{{\sgn}(\nu')
	\cdot \nu' d\nu'}{\widetilde{\kappa}^2(z,\nu')}= \frac{|\nu'|
	d\nu'}{\widetilde{\kappa}^2(z,\nu')} .
$$
Возвращаясь  к старому обозначению переменной $\nu$,
получаем
$$
A_{2,2}= \int \limits_{\gamma} \int
\limits^{2\pi}_{0} \int \limits^{1}_{-1} R(z,-\nu)
|f(z,\omega(\nu,\varphi),t)|^+ |\nu| d \nu d \varphi  d s_z,
$$
$$
A_{2,3} = \int \limits_{\gamma}
\int
\limits^{2\pi}_{0} \int \limits^{1}_{-1} 
T(z,\psi^{-1}(z,\nu)) |f(z,\omega(\nu,\varphi),t)|^+ |\nu| d \nu d
\varphi  d s_z.
$$
Учитывая соотношения
$$
\kappa(z,\nu)=1/\kappa(z,-\nu), \quad \psi^{-1}(z,\nu)=-\psi(z,-\nu),
\quad \text{при} \; 1-\kappa^2(z,\nu)(1-\nu^2) \geq 0,
$$
\begin{multline}
R_{\|}(z,-\nu)=\frac{\widetilde{\kappa}(z,-\nu)
	\psi(z,-\nu)+\nu}{\widetilde{\kappa}(z,-\nu) \psi(z,-\nu)-\nu}= \frac{
	-\psi^{-1}(z,\nu)/\widetilde{\kappa}(z,\nu)+\nu}{-\psi^{-1}(z,\nu)/\widetilde{\kappa}(z,\nu)
	-\nu}= \\ =-\frac{\nu \widetilde{\kappa}(z,\nu) - \psi^{-1}(z,\nu)}
{\nu\widetilde{\kappa}(z,\nu) + \psi^{-1}(z,\nu)}=
-R_{\|}(\psi^{-1}(z,\nu)), \nonumber
\end{multline}
\begin{multline}
R_{\bot}(z,-\nu)=\frac{\psi(z,-\nu)-\widetilde{\kappa}(-z,\nu)\nu}
{\psi(z,-\nu)-\widetilde{\kappa}(z,-\nu)\nu}= \frac{-
	\psi^{-1}(z,\nu)+\nu/\widetilde{\kappa}(z,\nu)}{-
	\psi^{-1}(z,\nu)-\nu/\widetilde{\kappa}(z,\nu)}= \\ = -\frac{ \nu -
	\psi^{-1}(z,\nu)\widetilde{\kappa}(z,\nu)}{\nu+
	\psi^{-1}(z,\nu)\widetilde{\kappa}(z,\nu)} =-R_{\bot}(\psi^{-1}(z,\nu)), \nonumber
\end{multline}
\begin{multline}
R(z,-\nu)=0.5\{R^2_{\|}(z,-\nu) +R^2_{\bot}(z,-\nu)\}= \\=
0.5\{R^2_{\|}(z,\psi^{-1}(z,\nu)) +R^2_{\bot}(z,\psi^{-1}(z,\nu))\}=
R(\psi^{-1}(z,\nu)), \nonumber
\end{multline}
$$
T(\psi^{-1}(z,\nu))=1-R(z,\psi^{-1}(z,\nu))=1-R(z,-\nu)=T(z,-\nu),
$$
из (29) находим
\begin{multline}
A_2 \geq  A_{2,1}- A_{2,2} - A_{2,3}=  \int \limits_{\gamma} \int \limits_{\Omega}| f
(z,\omega,t)|^+|\nu | d\omega d s_z - 
\\-
\int \limits_{\gamma} \int
\limits^{2\pi}_{0} \int \limits^{1}_{-1} R(z,-\nu)
|f(z,\omega(\nu,\varphi),t)|^+ |\nu| d \nu d \varphi  d s_z - \\-
\int \limits_{\gamma}
\int
\limits^{2\pi}_{0} \int \limits^{1}_{-1} 
T(z.-\nu) |f(z,\omega(\nu,\varphi),t)|^+ |\nu| d \nu d
\varphi  d s_z=0.
\end{multline}
Из (26),(27) следует, что $A=A_1+A_2+A_3 \geq A_3$ при почти всех
$t\in [0,\infty)$. Согласно (25) функция $A=0$, следовательно, и
$A_3=0$ при почти всех $t\in [0,\infty)$. Так как функция $\mu(r) > 0$,
поэтому и $f=0$ почти всюду на $G \times \Omega \times
[0,\infty)$. 


Лемма доказана. 
$\blacksquare$


\begin{lemma}
	Решение уравнения
	\begin{equation}
	f= \mathcal P (h+\mathcal B f^+) + \mathcal E \Phi
	\end{equation}
	существует, единственно и справедлива следующая оценка
	\begin{equation}
	\|f\|_{L_{\infty}(X)} \leq \max \left \{ \| h \|_{L_{\infty}(X^-_{ext})}, \left
	\|\frac{\Phi}{\mu} \right\|_{L_{\infty}(X)}\right\},
	\end{equation}
\end{lemma}
{\bf Доказательство.} Единственность решения уравнения (31)
вытекает из леммы 1, покажем существование решения уравнения (31).

Обозначим через $\Phi_{\pm},h_{\pm}$ функции следующего вида:
$$
\Phi_{+}(r,\omega,t)=\left\{%
\begin{array}{ll}
\Phi(r,\omega,t), & \hbox{если}\; \Phi(r,\omega,t)\geq 0;\\
0, & \hbox{иначе;} \\
\end{array}%
\right.
\Phi_{-}(r,\omega,t)=\left\{%
\begin{array}{ll}
\Phi(r,\omega,t), & \hbox{если} \;\Phi(r,\omega,t)\leq 0; \\
0, & \hbox{иначе,} \\
\end{array}%
\right.
$$
$$
h_{+}(z,\omega,t)=\left\{%
\begin{array}{ll}
h(z,\omega,t), & \hbox{если}\; h(z,\omega,t)\geq 0;\\
0, & \hbox{иначе;} \\
\end{array}%
\right.
h_{-}(z,\omega,t)=\left\{%
\begin{array}{ll}
h(z,\omega,t), & \hbox{если} \;h(z,\omega,t)\leq 0; \\
0, & \hbox{иначе.} \\
\end{array}%
\right.
$$
Положим $f_{\pm,0}={\mathcal P} h_{\pm} + {\mathcal E}
\Phi_{\pm}$ и построим итерационный процесс
\begin{equation}
f_{\pm,n}= \mathcal P \mathcal B f^+_{\pm,n-1} + f_{\pm,0} \quad
n=1,2,....
\end{equation}
Так как операторы $\mathcal P,\mathcal E, \mathcal B$
неотрицательные, то $f_{+,0}, f_{+,1},..., f_{+,n},...$ ---
монотонно возрастающая последовательность функций, а $f_{-,0},
f_{-,1},..., f_{-,n},...$ --- монотонно убывающая. Покажем, что
последовательность $\{f_{+,n} \}$ ограничена сверху, а
последовательность $\{f_{-,n} \}$ --- снизу.

Для функции $f_{+,0}$ оценка (32) вытекает из следующей цепочки
неравенств, справедливой при почти всех $(r,\omega,t)\in X$,
\begin{multline}
f_{+,0} \leq \max \limits_{1\leq i \leq m} \left \|
h_{+}(r-d(r,-\omega,t)\omega,\omega,t-d(r,-\omega,t)/v_i) \exp \left(-
\mu_i d(r,-\omega,t)\right) + \right.
\\
\left. +
\int \limits_0^{d(r,-\omega,t)} \exp \left(- \mu_i \tau
\right)
\Phi_+(r-\tau\omega,\omega,t-\tau/v_i) d\tau \right \|_{L_{\infty}(G_i \times \Omega \times [0,\infty))}
\leq \\
\leq \max \limits_{1\leq i \leq m} \left \| \exp (- \mu_i
d(r,-\omega,t) \|h\|_{X^-_{ext}} + (1- \exp (- \mu_i
d(r,-\omega,t)) \left\|\frac{\Phi_+}{\mu} \right\|_{X}
\right\|_{G_i \times \Omega \times [0,\infty)} \leq \\ \leq \max
\left\{\|h\|_{X^-_{ext}}, \left\|\frac{\Phi_+}{\mu} \right\|_{L_{\infty}(X)}
\right\}.
\end{multline}
Предполагая, что для функции $f_{+,n-1}$ неравенство (32)
выполнено, убедимся, что оно справедливо и для функции $f_{+,n}$.
Действительно, так как носители функций $h$ и ${\mathcal B} f^+_{+,n}$
не пересекаются и $\|{\mathcal B}\| \leq 1$ , то при почти всех
$(r,\omega,t)\in X$ из (34) получаем
\begin{multline}
f_{+,n}(r,\omega,t) \leq \max \limits _{1\leq i \leq m} \left\|
\max \left\{\|h\|_{L_{\infty}(X^-_{ext})}, \|\mathcal B f^+_{+,n-1}\|_{L_{\infty}(X^-)}
\right\} \exp (- \mu_i d(r,-\omega,t) + \right.
\\
\left.
+ \left\|\frac{\Phi_+}{\mu} \right\|_{L_{\infty}(X)}
\left(1-\exp (- \mu_i d(r,-\omega,t)) \right) \right\|_{L_{\infty}(G_i \times
	\Omega \times [0,\infty))} \leq \\ \leq
\max \limits _{1\leq i \leq m} \left\|
\max \left\{\|h\|_{L_{\infty}(X^-_{ext})}, \max \left\{\|h\|_{L_{\infty}(X^-_{ext})},
\left\|\frac{\Phi_+}{\mu} \right\|_{L_{\infty}(X)} \right\} \right\} \exp (-
\mu_i d(r,-\omega,t) + \right.
\\
\left.
+ \left\|\frac{\Phi_+}{\mu} \right\|_{L_{\infty}(X)}
\left(1-\exp (- \mu_i d(r,-\omega,t)) \right) \right\|_{G_i \times
	\Omega \times [0,\infty)} \leq \max \left\{\|h\|_{L_{\infty}(X^-_{ext})},
\left\|\frac{\Phi}{\mu} \right\|_{L_{\infty}(X)} \right\}.
\end{multline}
Таким образом, монотонная последовательность $f_{+,0},
f_{+,1},..., f_{+,n},...$ ограничена сверху, следовательно она
имеет предел $f_+=\lim \limits_{n \to \infty} f_{+,n}$ в
пространстве $L_{\infty} (X)$. Аналогично показывается, что
$$
f_{-,n}(r,\omega,t) \geq -\max \left\{\|h\|_{L_{\infty}(X^-_{ext})},
\left\|\frac{\Phi}{\mu} \right\|_{L_{\infty}(X )}\right\}
$$
при почти всех $(r,\omega,t)\in X$.

Поскольку для функции $f\in D$ соотношение (33) также справедливо
для $f^+_{\pm,n}$, то повторяя рассуждения, нетрудно убедиться в
существовании предельных функций $f^+_{\pm} \in L_{\infty}(X^+)$.


Переходя к пределу в равенстве (33), приходим к выводу, что
функции $f_{\pm}$ удовлетворяют уравнению (31), при
$\Phi=\Phi_{\pm}$. Таким образом, мы показали существование
решения $f$ уравнения (31), удовлетворяющего оценке (32).

$\blacksquare$

Из лемм 1,2, вытекает, что оператор обратный к оператору ${\cal L}$
существует и ограничен. Введем на линейном множестве $D$ норму
\begin{equation}
\|f\|_{D}= \max \left \{\|f^-\|_{L_{\infty}(X^-_{ext})},\left \|\frac{{\cal L}
	f}{\mu} \right\|_{L_{\infty}(X)}\right\}.
\end{equation}
Так как $\mu \geq conts >0$, то из неравенства (32) вытекает
\begin{equation}
\|f\|_{L_{\infty}(X)} \leq \|f\|_{D}.
\end{equation}
Следовательно, сходимость последовательности функций по норме $D$
влечет за собой сходимость в пространствах $L_{\infty}(X^+)$ и
$L_{\infty}(X)$. Из этого вытекает, что множество $D \subset
L_{\infty}(X)$ с нормой (36) образует банахово пространство
функций.




\begin{theorem}
	Пусть
	\begin{equation}
	\overline{\lambda} = \max \limits_{1\leq i\leq m} \lambda_i
	<1\,\quad \lambda_i = \sigma_i/\mu_i,
	\end{equation}
	тогда решение $I$ уравнения (13) в $D$ существует и единственно.
\end{theorem}
{\bf Доказательство.} Поскольку функция $p$ неотрицательна и
удовлетворяет условию нормировки и, кроме того, справедливо
неравенство (37), то для $\|\mathcal S I/\mu \|$ получаем
\begin{equation}
\left\|\frac{\mathcal S I}{\mu} \right \|_{L_{\infty}(X)}= \left\|\frac{
	\sigma(r)}{\mu(r)} \int \limits_{\Omega} p(r,\omega \cdot \omega')
I(r,\omega',t)d\omega' \right \|_{L_{\infty}(X)} \leq \overline{\lambda}\|I\|_{L_{\infty}(X)}
\leq \overline{\lambda} \|I\|_{D}.
\end{equation}
По построению $\mathcal L (\mathcal P \mathcal B + \mathcal E
\mathcal S) I = \mathcal S I$, и $(\mathcal P \mathcal B I +
\mathcal E \mathcal S I)^- =0$ на $X^-_{ext}$, следовательно, из
леммы 2 находим
$$
\|(\mathcal P \mathcal B + \mathcal E \mathcal S) I\|_{D} = \max
\left\{ 0, \left \| \frac{\mathcal S I}{\mu} \right \|_{L_{\infty}(X)}
\right\} \leq \overline{\lambda} \|I\|_{D}.
$$
Из (39) и последнего неравенства вытекает, что
$$
\|\mathcal P \mathcal B + \mathcal E \mathcal S\|_{D \to D} \leq
\overline{\lambda} <1.
$$
Так как норма оператора $\mathcal P \mathcal B + \mathcal E
\mathcal S$, действующего в банаховом пространстве $\mathcal{D}$,
меньше единицы, то уравнение (13) при выполнении условия (38)
однозначно разрешимо, и решение может быть найдено методом
последовательных приближений:
$$
I_n =(\mathcal P \mathcal B + \mathcal E \mathcal S)I_{n-1} +I_0,
\quad n=1,2,....
$$
Доказательство теоремы 1 завершено.$\blacksquare$

Отметим, что условие (38) обеспечивает существование и
единственность ограниченного решения при $t \in [0,\infty)$. В 
классе функций, допускающих неограниченный рост решения при $t
\to \infty$, или в случае конечного промежутка времени ($t\in [0,t^*], t^* <
\infty $) выполнение условия $\overline{\lambda}<1$ для корректности задачи 
не требуется.  Действительно, если $\sigma_i \geq \mu_i$, то после замены $I= I_{\lambda} e^{\lambda t}$
где $\lambda$ произвольное число, удовлетворяющее условиям
$\dfrac{\sigma_i}{\mu_i +\lambda/v_i} < 1,\, i=1,2,...,m$,
исходные данные начально-краевой задачи для функции $I_{\lambda}$
будут автоматически удовлетворять условию (38). Из теоремы 1 вытекает
существование и единственность решения $I_{\lambda}$, а
следовательно, и решения $I$ исходной задачи. 

\section{Метод  Монте-Карло для нахождения решения начально-краевой задачи}
Существует большое разнообразие численных методов решения
уравнений переноса излучения \cite{1,2,3,4,5,12,13,14,15,16,17,18,19}. Тем не менее, в
общем  многомерном случае и в особенности при решении реальных
задач методам Монте-Карло альтернативы практически нет
\cite{34}. 
Представление в виде усеченного ряда Неймана
\begin{equation}
I_N = \sum_{n=0}^{N}\ (\mathcal{PB} + \mathcal{ES})^n (\mathcal P h
+ \mathcal E J)
\end{equation}
и его рекуррентный аналог
\begin{equation}
I_0 = \mathcal P h + \mathcal E J,
\end{equation}
\begin{equation}
I_n = (\mathcal{PB} + \mathcal{ES}) I_{n-1} + I_0,\quad n = 1, 2 \dots N
\end{equation}
являются основой для построения численных алгоритмов вычисления
искомой функции~$I$. Представление искомой функции в таком виде
имеет наглядное физическое представление. Каждый член
последовательности $I_n$ является аппроксимацией функции $I$, 
с учётом излучения, про взаимодействовавшего со средой не более, чем $n$ раз.
Применение оператора $\mathcal{PB} + \mathcal{ES}$ к   элементу
последовательности $I_{n-1}$ определяет вклад излучения,
испытавшего от $1$  до    $n$  актов взаимодействия со средой, а
слагаемое $I_0$ учитывает вклад излучения, пришедшего от
источников без взаимодействия со средой.

Согласно методу Монте-Карло, приближение $I_N$
вычисляется как среднее значение выборки некоторого объема $M$ для случайной 
величины ${\cal I}_N$, с математическим ожиданием равным $I_N$, то есть
\begin{equation}
I_N \approx \overline{\cal I}_{N}.
\end{equation}

Для нахождения значений ${\cal I}_{n}$ в точке
$x=(r,\omega,t)\in G_i \times \Omega \times [0,\infty)$
вычисляется функция ${\cal I}_{n-1}$ в четырёх других точках:
\begin{equation}
\begin{array}{ll}
  x_{re}=(r_h,\omega_{re}, t_h), &
  \begin{array}{l}
    r_h = r - d(r,-\omega,t)\omega,\\
    \omega_{re} = \omega - 2\nu n(r_h),  \quad  \nu=n(r_h) \cdot \omega,\\
    t_h = t - d(r,-\omega,t)/v_i, \\
  \end{array}%
\\
\\
  x_{tr}=(r_h,\omega_{tr},t_h), &
  \begin{array}{l}
    \omega_{tr}=\psi (\nu)
    n(r_h) + \widetilde{\kappa}(\nu)( \omega -  \nu n(r_h)),
  \end{array}
\\
\\
  x_{di}=(r_{h},\omega_{di},t_{h}), &
\\
\\
  x_{sc}=(r_{sc},\omega_{sc},t_{sc}), &
  \begin{array}{l}
    r_{sc}=r-\tau \omega, \quad
    t_{sc}=t-\tau/v_i.
  \end{array}
\end{array}
\end{equation}

Кроме того, вычисляются функции $h$ и $J$ в точках
$$
x_h=(r_h,\omega,t_h),\quad
x_s=(r_{sc},\omega,t_{sc})
$$
и полагается:
\begin{multline}
{\cal I}_n(x) = \exp(-\mu_i d(r, -\omega, t)) 
\\ (h(x_h) +  \alpha_f (x_h)(R(x_h) {\cal I}_{n-1}^+
(x_{re}) + T(x_h) {\cal I}_{n-1}^+ (x_{tr})) + \frac{\alpha_d(x_h) K_d(x_h)} {f_{\omega_{di}}} {\cal I}_{n-1}^+ (x_{di})) +
\\ + \frac{1 - \exp(-\mu_i d(r, -\omega, t))}{\mu_i} \left(J(x_s) + \frac{\sigma_i p_i(\omega \cdot \omega_{sc})} { f_{\omega_{sc}}} {\cal I}_{n-1}(x_{sc})\right),
\end{multline}


\begin{equation}
{\cal I}_0(x) = \exp(-\mu_i d(r, -\omega, t)) h(x_h) +
\frac{1 - \exp(-\mu_i d(r, -\omega, t))}{\mu_i} J(x_s),
\end{equation}
где
$$
K_d(x_{di}) = \left \{ 
\begin{array}{ll} 
	\dfrac{|n(r_h)\cdot \omega_{di}|}{\pi} T_d(x_{di}) & w_{di} \in \Omega(r_h,-\omega), \\
	\dfrac{|n(r_h)\cdot \omega_{di}|}{\pi} R_d(x_{di}) & w_{di} \in \Omega(r_h,-\omega).
\end{array} \right.
$$
Случайная величина $\tau$ распределенна с плотностью вероятности
$$
f_\tau = \displaystyle{\frac{\exp(-\mu_i \tau)}{(1-\exp(-\mu_i d(r,
-\omega, t)))}}.
$$
А вот для розыгрыша случайных векторов $\omega_{sc}, \omega_{di}$ 
попытаемся использовать законы распределения, 
минимизирующие дисперсию получаемой оценки.
Случайные величины  $\omega_{di}$, $\tau, \omega_{sc}$ независимые, а значит
\begin{multline}
D({\cal I}_n(x)) = \exp(-\mu_i d(r, -\omega, t)) \frac{\alpha_d(x_h) K_d(x_h)} {f_{\omega_{di}}} D({\cal I}_{n-1}^+ (x_{di})) + \\
 + \frac{1 - \exp(-\mu_i d(r, -\omega, t))}{\mu_i p(\omega_{sc})} D(J(x_{s}) + \sigma_i {\cal I}_{n-1}(x_{sc}))
\end{multline}

Часто в качестве плотностей распределения $\omega_{di}, \omega_{sc}$ используют следующие функции: 
${\displaystyle f_{\omega_{di}} = K_d(x_h)}$, 
${\displaystyle f_{\omega_{sc}} = p_i(\omega \cdot \omega_{sc}) }$ \cite{33}.
Такой выбор является оптимальным в случае, 
когда функция ${\cal I}_{n-1}$ является постоянной в исследуемой области
\begin{equation}
{\cal I}_{n-1}(x) = const.
\end{equation}

Действительно,
\begin{equation}
{\overline{\cal I}_{n-1}(x_{di})} = 
\int\limits_{\Omega} K_d(x_h) {\cal I}_{n-1}(r_h, \omega_{di}, t_h)d\omega_{di},
\end{equation}

\begin{equation}
{\overline{\cal I}_{n-1}(x_{sc})} = 
\int \limits^{d(r,-\omega,t)}_0 \int\limits_{\Omega} \exp(- \mu_i \tau) 
p(r,\omega \cdot\omega_{sc}) {\cal I}_{n-1}(r-\tau\omega,\omega_{sc},t-\tau/v_i)d\omega_{sc}d\tau ,
\end{equation}
а следовательно плотности распределения $x_{di}, x_{sc}$, с точностью до нормирующего множителя, равны 
соответствующим подынтегральным функциям,
что в свою очередь является критерием минимальности дисперсии с.в. вида (49), (50).
Аналогичное замечание справедливо и для с.в. $J(x_s)$.

 Расчётная формула в случае такого выбора функций $f_{\omega_{di}}, f_{\omega_{sc}}$ 
также упрощается и принимают вид:
 \begin{multline}
{\cal I}_n(x) = \exp(-\mu_i d(r, -\omega, t)) 
\\ (h(x_h) +  \alpha_d(x_h)(R(x_h) {\cal I}_{n-1}^+
(x_{re}) + T(x_h) ) {\cal I}_{n-1}^+ (x_{tr}) + \alpha_d(x_h) {\cal I}_{n-1}^+ (x_{di})) +
\\ + \frac{1 - \exp(-\mu_i d(r, -\omega, t))}{\mu_i} \left(J(x_s) + \sigma_i {\cal I}_{n-1}(x_{sc})\right),
\end{multline}
{Отметим также, что предположение (48), в некотором смысле, 
выражает отсутствие априорной информации о поведении функции ${\cal I}_{n-1}$ 
в момент вычисления значения ${\cal I}_n(x)$.

В случае, когда функция плотности внешних источников $h(x)$ является заранее известной,
мы бы могли использовать эту информацию для уменьшения дисперсии генерируемых с.в.

Рассмотрим ветвящиеся цепочки случайных величин, функции перехода которых задаются соотношениями (44).
Введём обозначения для элементов этих цепочек по следующему рекурсивному правилу:
элемент цепочки $N$-го уровня обозначим за $x_{in_1,in_2, ..., in_n}$, 
если он был получен из элемента $x_{in_1,in_2, ..., in_{n-1}}$ после взаимодействия типа $in_n$,
где $in_i \in \{tr, re, sc, di, h, s\}$.

Заменим в (49), (50) ${\cal I}_{n-1}(x)$ на его определение из формул (41), (42):
\begin{equation}
{\overline{\cal I}_{n-1}(x_{di})} = 
{\cal PB}_{d} I_{n-1} = {\cal PB}_{d} (\mathcal{P} h + \mathcal E J + (\mathcal{PB} + \mathcal{ES}) I_{n-2}),
\end{equation}
\begin{equation}
{\sigma_i \overline{\cal I}_{n-1}(x_{sc})} = {\cal ES}I_{n-1} = 
{\cal ES} (\mathcal{P} h + \mathcal E J + (\mathcal{PB} + \mathcal{ES}) I_{n-2}).
\end{equation}
Выпишем подробнее слагаемые, отвечающие за вклад излучения, пришедшего от источника после однократного взаимодествия со средой:
\begin{equation}
{\overline{\cal I}_{n-2}^{d, h}} = {\cal PB}_{d} (\mathcal{P} h) =
\exp(- \mu_i d(r_h, -\omega_{di}, t_h))  \int\limits_{\Omega} K_d(x_h) h(x_{di, h})d\omega_{di}
\end{equation}

\begin{multline}
{\overline{\cal I}_{n-2}^{sc, h}} = {\cal ES} (\mathcal{P} h) = 
\sigma(r) \int \limits^{d(r,-\omega,t)}_0 \int\limits_{\Omega} \exp(- \mu_i (d(r-\tau\omega, -\omega_{sc}, t-\tau/v_i) + \tau)) \\ 
p(r,\omega \cdot\omega_{sc}) h(x_{sc, h})d\tau d\omega_{sc},
\end{multline}

Минимум дисперсии таких с.в. достигается при 
\begin{equation}
f_{\omega_{di}} =\frac{K_d(x_h) h(x_{di, h})} {\int\limits_{\Omega} K_d(x_h) h(x_{di, h}) d \omega_{di}}
\end{equation}
\begin{equation}
f_{\omega_{sc}} = \frac{\exp(- \mu_i (d(r-\tau\omega, -\omega_{sc}, t-\tau/v_i) + \tau)) p(r,\omega \cdot\omega_{sc}) h(x_{sc, h})}
{\int\limits_{\Omega}\exp(- \mu_i (d(r-\tau\omega, -\omega_{sc}, t-\tau/v_i) + \tau)) p(r,\omega \cdot\omega_{sc}) h(x_{sc, h}) d \omega_{sc}}
\end{equation}

В случае, когда выбор законов распределения $f_{\omega_{di}}, \omega_{sc}$ 
влияет на дисперсию только соответствующих слагаемых
${\cal I}_{n-2}^{di, h}, {\cal I}_{n-2}^{sc, h}$, 
такой выбор оптимизирует розыгрыш и исходной с.в. ${\cal I}_N(x)$.
Допущение (48) на самом деле выражает отсутствие 
априорной информации о распределении значений функции ${\cal I}_{n-2}$.

\section{Вычислительные эксперименты}

Для проверки адекватности построенной модели процесса переноса излучения, 
а также с целью определения эффективности предложенного вычислительного метода для решения
задач визуализации трёхмерных нестационарных сцен были проведены вычислительные эксперименты,
в ходе которых рассчитывались перспективные проекции для заданных трёхмерных сцен в 
различные моменты времени. Сравнивая полученные результаты с естественными представлениями 
человека о распространении видимого света, мы получили возможность дать косвенную оценку
предложенных методов.  

Сделаем несколько замечаний относительно цветовой гаммы реконструируемых трехмерных сцен. 
Строго говоря, изучаемая модель не рассматривает вопросы зависимости 
распространения излучения от его частотных характеристик,  хотя полученные изображения и являются цветными.
Для получения цветного изображения мы проводили расчёты отдельно для каждого из RGB каналов,
используя при этом различные наборы оптических характеристик среды.
С одной стороны, такое совмещённое изображение даёт нам наглядное и естественное представление о результатах
многократного взаимодействия излучения со средой.  С другой стороны, благодаря цветовому контрасту,  на построенных изображениях гораздо  отчётливее наблюдается  пространственное  распределение ошибки, вызванной той или иной реализацией метода Монте-Карло. 

Перейдём к описанию самих экспериментов.
Целью первого эксперимента является визуализация эволюционного процесса распространения излучения в среде.
Исследуемая среда, расположена над диффузно отражающей плоскостью $(\alpha_f = 0, R_d(z) = 1)$.
При диффузном взаимодействии излучение ослабевает с коэффициентами $\alpha_d = (0.7, 0.2, 0.2)$,
для красного, зелёного и синего канала соответственно. 

На плоскости находятся:

несколько диффузно отражающих шаров с такими же граничными характеристиками, 
как и у самой плоскости; 

несколько  шаров, зеркально отражающих  излучение только зелёного канала  $(\alpha_f = (0, 1, 0),  \alpha_d = 0, \kappa(z) = 3)$.
Коэффициенты $\mu$ и $\sigma$ внутри зеркальных и диффузных шаров совпадают с аналогичными коэффициентами основной среды;

шары, с прозрачной поверхностью $(\alpha_f = 1, \alpha_d = 0, \kappa(z) = 1)$, 
заполненные \textit{сильно} рассеивающим веществом с коэффициентом взаимодействия  $\mu(r) = 2.3$ и
$\sigma(r)=(0.1, 0.7, 0.8)$ для каждого канала.


В центре сцены находится статуэтка \cite{35}, её поверхность имеет следующие характеристики: 
$\alpha_f = (0.05, 0.45, 0.05),  \alpha_d = (0.35, 0.1, 0.1),  R_d(z) = 1, \kappa(z) = 1.3$, $\mu(r) = 2.3$,
$\sigma(r)=(0.1, 0.7, 0.8)$.

Источник излучения находится на границе исследуемой области и его плотность экспоненциально убывает как по временной, так и по пространственным координатам
от точки $(z_s, t_s)$, которая смещена немного вправо и вверх от наблюдателя в пространстве 
и на некоторое время назад в прошлое во времени $h(z, \omega, t) = exp(-|z - z_s|^2 - |t - t_s|^2)$.

На рис. 1 представлены результаты работы алгоритма для моментов времени $t = 110, 115, 120 \dots$.
Мы можем видеть, как с увеличение временной координаты излучение последовательно
освещает всё более удалённые участки сцены.
Этот эффект, а также тени, полупрозрачность рассеивающих шаров, блики на зеркальных поверхностях,
отсутствие бликов на диффузных границах и взаимное переотражение объектов друг в друге говорят о
возможности получения точных, фотореалистичных реконструкций с использованием разработанных алгоритмов.

\begin{figure}[H]
	\foreach \x in {110,120,130,140,150,160,170,180}
	{ 
		\begin{subfigure}[b]{0.24\linewidth}
			\centering
			\includegraphics[width=0.9\linewidth]{nonstat/\x.png}
			\caption{\x}
		\end{subfigure}
	}
	\caption{Эволюционный процесс решения нестационарной задачи УПИ}
\end{figure}

Во втором эксперименте мы попытались сравнить эффективность 
алгоритмов с использованием оптимизированного розыгрыша с.в. $\omega_{di}, \omega_{sc}$ по формулам (55), (56) 
и стандартным методом, с использованием плотностей ${\displaystyle f_{\omega_{di}} = K_d(x_h)}$, 
${\displaystyle f_{\omega_{sc}} = p_i(\omega \cdot \omega_{sc}) }$.
Мы зафиксировали значение временной координаты $t = 155$ и сохраняли 
изображения через определённые промежутки времени работы алгоритма.
На рис. 2 приведены результаты расчетов  для двух модификаций алгоритма Монте-Карло.
Первая пара  --- изображения, полученные через $2^1$ сек., 
вторая --- через $2^3$ сек., третья --- через $2^5$ сек. и т.д.
Верхние рисунки соответствуют стандартному методу, нижние -- оптимизированному. 


Следует отметить, что при генерации оптимизированных с.в. вместо формул (55), (56) мы использовали
намного более простой закон, позволяющий существенно упростить генерацию необходимых с.в.:
$$
f_{\omega_{di}} = \frac{h(x_{di, h})} {\int\limits_{\Omega} h(x_{di, h}) d \omega_{di}}, 
f_{\omega_{sc}} = \frac{h(x_{sc, h})} {\int\limits_{\Omega} h(x_{sc, h}) d \omega_{sc}}.
$$

Таким образом, для выбранной функции распределения граничных источников
генерация сводится к розыгрышу с.в. $x_{di, h}, x_{sc, h}$ нормально распределённых на плоскости источника.

\begin{figure}[H]
	\foreach \x in {1,3,5,7,9,11,13,15}
	{ 
		\begin{subfigure}[b]{0.24\linewidth}
			\centering    
			\includegraphics[width=0.9\linewidth]{noptim_result/\x.png}
			\includegraphics[width=0.9\linewidth]{optimized_result/\x.png}
			\caption{$2^{\x}$ сек.}
		\end{subfigure}
	}
	\caption{Сравнения стандартного и оптимизированного методов розыгрыша с.в.}
\end{figure}

Анализируя полученные результаты  следует отметить, что, несмотря на достаточно грубые 
допущения, некоторый эффект оптимизации вычислений имеет место быть. На первых изображения качество изображений,
полученных оптимизированным методом, намного превосходит аналогичные результаты не оптимизированного метода.
Вместе с тем можно наблюдать и основные отрицательные стороны нашего метода. 
Прежде всего это синие и зелёные выбросы на диффузно отражающей поверхности, отсутствие отражения зеркальных шаров прямо под ними, плохую отрисовку фигур 
за сильно рассеивающей средой
и другие детали изображения, возникающие в результате многократного взаимодействия излучения со средой.
Эти недостатки являются прямым следствием идеи, заложенной в наш метод.
На каждом шаге вычислений мы используем функцию распределения 
источника излучения, в качестве аппроксимации решения с предыдущего шага.
Очевидно, что при таком подходе, излучение, многократно провзаимодействовавшее 
со средой, будет учтено менее точно, либо вовсе утрачено.

Проведённый анализ позволяет определить зону применимости каждого из методов.
Оптимизированный розыгрыш с.в. на этапе моделирования диффузно отражённого и рассеянного излучения
позволяет получать достаточно качественное изображение соответствующих областей изображения
в короткие сроки. Однако при средних сроках отрисовки, в случае наличия
на сцене достаточно сильного многократного провзаимодействовавшего излучения, использование
оптимизированного алгоритма может привести к появлению выбросов, при учёте такого излучения.
В пределе, при увеличении времени работы, оба метода дают схожие результаты.





\begin{thebibliography}{1}

\Bibitem{1} \by  S.~Chandrasekhar \book  Radiative transfer \publaddr London \publ Oxford University Press \yr 1950

\Bibitem{2} \by V.~S. Vladimirov  \paper Mathematical problems of the one-velocity theory of particle transport \jour  Transactions of the V.A. Steklov Mathematical Institute  \vol 61
\yr 1961 \pages 3--158


\Bibitem{3} \by K.~M.~Case, P.~F.~Zweifel \book Linear Transport Theory \publ 
Addison-Wesley. Co. Reading, MA \yr 1967


\Bibitem{4} \by C.~Cercignani \book Theory and Application of the Boltzmann Equation \publ Elsevier \publaddr New York \yr 1975

\RBibitem{5} \by  A.~Ishimaru \book  Wave Propagation and Scattering in 	Random Media \publaddr  New York \publ Academic Press \yr  1978

\bibitem{6} \by    D.~S.~Anikonov,   A.~E.~Kovtanyuk,    I.~V.~Prokhorov \book Transport
	Equation and Tomography \publaddr Utrecht-Boston \publ  VSP \yr 2002

\Bibitem{7} \by   I.~V.~Prokhorov,   I.~P.~Yarovenko, and T.~V.~Krasnikova \paper An extremum problem for the radiation transfer
equation \jour Journal of Inverse and Ill-Posed Problems \yr 2005
\vol 13 \issue 4 \pages 365--382

\Bibitem{8} \by A.~E.~Kovtanyuk, I.~V.~Prokhorov \paper Tomography problem for
the polarized-radiation transfer equation \jour  Journal of Inverse and Ill-Posed Problems \yr 2006 \vol 14 \issue 6 \pages 609--620


\Bibitem{9}
\by I.~V.~Prokhorov, I.~P.~Yarovenko, V.~G.~Nazarov
\paper Optical tomography problems at layered media
\jour Inverse Problems
\yr 2008
\vol 24
\issue 2
\papernumber 025019
\totalpages 13

\bibitem{10}
\by G.~Bal \paper Inverse transport theory and applications
\jour Inverse Problems \yr 2009  \vol 25 \issue 5 \papernumber 025019
\totalpages 13


\Bibitem{11}
\by I.~V.~Prokhorov, I.~P.~Yarovenko
\paper Analysis of the tomographic contrast during the immersion bleaching of layered biological tissues
\jour Quantum Electronics
\yr 2010
\vol 40
\issue 1
\pages 77--82

\Bibitem{12} 
\by
A.~Hussein, M.~M.~Selim
\paper Solution of the stochastic radiative transfer equation with Rayleigh scattering using RVT technique \jour 
Applied Mathematics and Computation \yr 2012 \vol 318 \issue 13  \pages  7193--7203


\Bibitem{13} 
\by A.~E.~Kovtanyuk, A.~Yu.~Chebotarev
\paper An iterative method for solving a complex heat transfer problem
\jour Applied Mathematics and Computation \vol 219 \issue 17 \yr 2013 \pages 9356--9362


\Bibitem{14} \by Yong Zhang, Hong-Liang Yi, He-Ping Tan  \paper Short-pulsed laser propagation in a participating slab with Fresnel
surfaces by lattice Boltzmann method \jour International Journal of Heat and Mass Transfer \vol 80 \yr 2015 \pages 717--726


\Bibitem{15} \by Lin-Feng Qian, Guo-Dong Shi, Yong Huang, Yu-Ming Xing \paper Backward and forward Monte Carlo method for vector radiative transfer in a two-dimensional graded index medium \jour Journal of Quantitative Spectroscopy and Radiative Transfer \yr 2017 \vol 200 \pages 225--233

\Bibitem{16} \by Zhen W., Shengcheng C., Jun Y., Haiyang G., Chao L., Zhibo Z.
\paper A novel hybrid scattering order-dependent variance
reduction method for Monte Carlo simulations of radiative transfer
in cloudy atmosphere \jour Journal of Quantitative Spectroscopy
and Radiative Transfer \vol 189 \yr 2017 \pages 283--302


\Bibitem{17} \by  Cleveland~M.A., Wollaber~A.B. \paper 
Corrected implicit Monte Carlo 
\jour Journal of Computational Physics \vol 359 \yr 2018 \pages 20-44



\Bibitem{18} 
\by
J.~Sun,  J.~A.~Eichholz 
\paper Splitting methods for differential approximations of the radiative transfer equation  \jour 
Applied Mathematics and Computation \yr 2018 \vol 322  \pages 140--150




\Bibitem{19} \by   I.~V.~Prokhorov \paper Boundary value problem of radiation transfer in an inhomogeneous medium with reflection conditions on the boundary
\jour Differential  Equation
\yr 2000
\vol 36
\issue 6
\pages 943--948



\Bibitem{20} \by   I.~V.~Prokhorov \paper 
On the solubility of the boundary-value problem of radiation transport theory with generalized conjugation conditions on the interfaces
\jour Izvestiya: Mathematics
\yr 2003
\vol 67
\issue 6
\pages 1243--1266
\

\Bibitem{21} \by I.~V.~Prokhorov \paper On the Structure of the Continuity Set of the Solution to a Boundary-Value Problem for the Radiation Transfer Equation \jour Mathematical Notes \yr 2009 \vol 86 \issue 2 \pages 234–-248



\Bibitem{22} \by A.~E.~Kovtanyuk, I.~V.~Prokhorov \paper A
boundary-value problem for the polarized-radiation transfer
equation with Fresnel interface conditions for a layered medium
\jour Journal of Computational and Applied Mathematics \yr 2011
\vol 235 \issue 8 \pages 2006--2014


\bibitem{23} \by A.A.~Amosov \paper Boundary value problem for the radiation transfer equation with
reflection and refraction conditions \jour Journal of Mathematical
Sciences  \vol 191 \issue 2 \yr 2013 \pages 101-149

\bibitem{24} \by A.A.~Amosov \paper Boundary Value Problem for the Radiation Transfer Equation with
Diffuse Reflection and Refraction Conditions \jour Journal of
Mathematical Sciences  \vol 193 \issue 2 \yr 2013 \pages 151-176

\bibitem{25} \by  A.~Amosov,  M.~Shumarov \paper
Boundary value problem for radiation transfer equation in multilayered medium with reflection and refraction conditions
\jour Applicable Analysis  \vol 95  \issue  7 \yr 2016

\bibitem{26}  \by A.A.~Amosov \paper Radiative Transfer Equation with Fresnel Reflection and Refraction Conditions in a System of Bodies with Piecewise Smooth Boundaries
\yr 2016 \jour Journal of Mathematical Sciences \vol 219 \issue 6
\pages 3--29





\Bibitem{27} \by I.~V.~Prokhorov  \paper Solvability of the initial-boundary value problem for an integrodifferential equation
\jour Siberian Mathematical Journal \yr 2012 \vol  53 \issue 2
\pages 301-309


\Bibitem{28} \by \by I.~V.~Prokhorov  \paper  The cauchy problem for the radiative transfer equation with generalized conjugation conditions \jour Computational Mathematics and Mathematical Physics \yr 2013 \vol 53
\issue 5 \pages 588--600

\Bibitem{29}
\by I.~V.~Prokhorov, A.~A.~Sushchenko
\paper On the well-posedness of the Cauchy problem for the equation of radiative transfer with Fresnel matching conditions \jour   Siberian Mathematical Journal 
\yr 2015
\vol 56
\issue 4
\pages 736--745



\Bibitem{30}
\by  I.~V.~Prokhorov, A.~A.~Sushchenko, A.~Kim \paper Initial boundary value problem for the radiative transfer equation with diffusion matching conditions\jour
Journal of Applied and Industrial Mathematics \yr 2017 \vol 11 \issue 1 \pages 115--124


\Bibitem{31}
\by  A.\,A.~Amosov \paper Initial-Boundary Value Problem for the Non-Stationary Radiative Transfer Equation with Fresnel Reflection and Refraction Conditions \jour Journal of Mathematical Sciences \vol 231 \issue 3 \yr  2018  \pages  279--337

\Bibitem{32}
\by  A.\,A.~Amosov \paper Initial-Boundary Value Problem for the Nonstationary Radiative Transfer Equation with Diffuse Reflection and Refraction Conditions  \jour Journal of Mathematical Sciences \vol 235 \issue 2 \yr  2018  \pages 117--137 


\Bibitem{33}
\by A.~Kim, I.~V.~Prokhorov \paper Theoretical and Numerical Analysis of an Initial-Boundary Value Problem for the Radiative Transfer Equation with Fresnel Matching Conditions \jour Computational Mathematics and Mathematical Physics \vol 58 \issue 5 \yr 2018 \pages 735--749


\Bibitem{34} 
\by  S.~M.~Prigarin,  T.~V.~Aleshina \paper Monte Carlo simulation of ring-shaped returns for CCD LIDAR systems \jour Russian Journal of Numerical Analysis and Mathematical Modelling \yr 2015 \vol 30 \issue 4 \pages 251-–257


\Bibitem{35} The Stanford 3D Scanning Repository \url{http://graphics.stanford.edu/data/3Dscanrep/}
\end{thebibliography}

\end{document}